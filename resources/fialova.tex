\documentclass[12pt, a4paper]{article}

\usepackage{xcolor}
\usepackage{amsthm}
\usepackage{amsmath}
\usepackage[czech]{babel} 

\theoremstyle{plain}
\newtheorem{veta}{Věta}
\newtheorem{lemma}[veta]{Lemma}
\newtheorem{tvrz}[veta]{Tvrzení}

\theoremstyle{plain}
\newtheorem{definice}{Definice}
\newtheorem*{pozor}{Pozorování}
\newtheorem*{cvic}{Cvičení}
\newtheorem*{fakt}{Fakt}

\theoremstyle{remark}
\newtheorem*{dusl}{Důsledek}
\newtheorem*{pozn}{Poznámka}
\newtheorem*{prikl}{Příklad}

\newenvironment{dukaz}{
	\par\medskip\noindent
	\textit{Důkaz}.
}{
	\newline
	\rightline{$\qedsymbol$}
}

\newcommand{\fialova}[1]{\textcolor{violet}{fialov#1}}

\title{Proč je \fialova{á} nejhorší barva?}
\author{Tomáš Turek}
\date{\today}

\begin{document}
	\maketitle
	
	\begin{center}
		{\small \textcolor{red}{Upozornění! Tento text je pouze satirický a ukazuje na nelogické argumenty jisté části lidí ve společnosti. Text slouží pouze k pobavení a všechny podobnosti s reálnými osobami jsou čistě náhodné.}}
	\end{center}
	
	V tomto textu si postupně vybudujeme teorii kolem \fialova{é} barvy a proč je centrem všeho dění. A to i když to tak nemusí na první pohled vypadat. Nejdříve začneme s definicí.
	
	Vzhledem k původu této teorie a terminologie smíme používat jisté ne zcela logické kroky. Někdo znalý výrokové a predikátové logiky by tyto postupy mohl označit za nelegální.
	
	\begin{definice}[\fialova{á}]
		Fialová odpovídá jménu premiéra a tím pádem přímo navazuje na práci celé vlády.
	\end{definice}
	
	\begin{pozor}
		Za vše co se odehraje v poslanecké sněmovně může vláda, protože je středobodem celé sněmovny.
	\end{pozor}
	
	Tohle pozorování by pro čtenáře mělo být jednoduché vidět. Nicméně si ukážeme už o něco silnější lemma.
	
	\begin{lemma}[české]
		Vše co se ve státě děje souvisí s vládou a tím pádem za to může \fialova{á}.
	\end{lemma}
	
	\begin{dukaz}
		Pro zkušené voliče SPD a ANO (popřípadě více divných stran) toto tvrzení je samozřejmostí. Nicméně i tak si ho raději popíšeme. Nechtť $U$ je nějaké dění ve společnosti. Protože lidé neřeší nic jiného než výši pěnez a luxu života, tak $U$ musí být socio-ekonomická záležitost. Protože za ekonomiku státu a výši minimálních mezd odpovídá vláda, tak nutně \fialova{á} je odpovědná barva. Pokud bychom brali nějaké jiné $U$, tak nejspíše chápete tento text.
	\end{dukaz}
	
	Tohle už je zajímavější tvrzení, ze kterého plyne jeden krásný důsledek.
	
	\begin{dusl}
		Můj život je neuspokující kvůli \fialova{é}.
	\end{dusl}
	
	\begin{proof}
		Jednoduše plyne z toho, že jsme součástí tohoto státu.
	\end{proof}
	
	Nyní si ukážeme další lemma, které zvětšuje plochu ovlivnitelnosti \fialova{é}.
	
	\begin{lemma}[EU]
		Za všechny události v rámci EU zodpovídá \fialova{á}.
	\end{lemma}
	
	\begin{dukaz}
		Protože náš stát je součástí EU a také vláda propaguje pro-evropské mínění, tak jsme velmi závislý na EU. Stejně tak \fialova{á} podporuje migrační kvóty a \textcolor{teal}{green deal} a má v europarlamentu zastoupení. Tedy přímo ovlivňuje dění v EU. V jiném případě nedělají dost a měli by v Bruselu více makať.
	\end{dukaz}
	
	Už se blížím k samotnému závěru a cílu tohoto textu, nicméně ještě předtím si řekneme jeden důsledek a poté musíme vyslovit jednou poslední lemma.
	
	\begin{dusl}
		\textcolor{violet}{Fialová} mi nutí elektrické auto a úsporné nesmysli.
	\end{dusl}
	
	\begin{lemma}["amerika" (čteno americký mpřízvukem)]
		Veškeré akce spojené s amerikou jsou přímo spojené s \fialova{ou}.
	\end{lemma}
	
	\begin{dukaz}
		Tento důkaz už je jednoduchý. Jelikož Amerika je spojencem naprosté většiny evropských států a tím pádem i evropy jako takové, tak EU má vliv na veškeré akce v americe. Dále můžeme použít předchozí lemma a vše je dokázáno.
	\end{dukaz}
	
	\begin{veta}
		Za vše může \fialova{á}.
	\end{veta}
	
	\begin{dukaz}
		Protože už jsme si vytvořili celou teorii, tak důkaz nebude nijak těžký. Vše co se ve světě děje se děje v nějakém státu. Potom $\forall \text{ stát} \in \text{Svět}$ má jistý vztah s amerikou, tak nutně její akce jsou s amerikou propojené. Díky poslednímu lemma už víme. že za to může \fialova{á}.
	\end{dukaz}
	
	\begin{dusl}
		\textcolor{violet}{Fialová} je nejhorší barva.
	\end{dusl}
	
	\begin{center}
		{\small \textcolor{red}{Děkuji pokud jste text dočetli až do konce a snad jste se aspoň trochu pobavili. Pokud vám příjde, že jste narazili na chyby v logice, tak to jste je nejspíše našli správně. Protože skoro všechny mezikroky jsou špatně a nedávají smysl. Pokud vám to \textit{ftipné} nepřišlo, tak máte asi rádi písničku \textit{désolé}.}}
	\end{center}
	
\end{document}