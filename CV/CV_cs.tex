\documentclass{article}

\usepackage{geometry}
\geometry{a4paper,total={190mm,257mm},left=10mm,top=15mm}

\usepackage{hyperref}
\usepackage{xcolor}
\hypersetup{colorlinks=true,urlcolor=blue}
\usepackage{tablefootnote}

\pagestyle{empty}

\begin{document}

\begin{center}
	\section*{{\huge Curriculum vitae}}
\end{center}

\hfill

\begin{minipage}[c]{.45\textwidth}
	\subsection*{Osobní informace}
	\begin{itemize}
		\item Jméno: \textit{Tomáš Turek}
		\item Narození: \textit{24.03.2001}
		\item Původ: \textit{Olomouc, Česká republika}
		\item Mail: \href{mailto:tom.turek@atlas.cz}{\textit{tom.turek@atlas.cz}}
		\item Github: \href{https://github.com/metury}{\textit{metury}}
		\item Řidíčský průkaz: \textit{B}.
	\end{itemize}
\end{minipage}
\hfill
\begin{minipage}[c]{.5\textwidth}
	\subsection*{Technologie a programovací jazyky}
	\begin{itemize}
		\item Pokročilá znalost: \textit{C++} a \textit{Qt}.
		\item Průměrná znalost: \textit{Java}, \textit{Python} a \textit{Rust}.
		\item Základní znalost: \textit{C\#}, \textit{Haskell} a \textit{Perl}.
	\end{itemize}
	
	\subsection*{Jazyky}
	\begin{itemize}
		\item Pokročilá znalost: \textit{Čeština}, \textit{Angličtina}.
		\item Základní znalost: \textit{Němčina}.
	\end{itemize}
\end{minipage}

\subsection*{Studium}

\begin{table}[!ht]\centering
	\begin{tabular}{p{20mm} p{50mm} p{80mm} p{20mm}}
		\textit{Roky} & \textit{Typ studia} & \textit{Škola} & \textit{Zakončení} \\
		2023-2025\tablefootnote{Plánované ukončení studia.} & Diskrétní modely a algoritmy & \href{https://www.mff.cuni.cz/}{Univerzita Karlova, Matematicko-Fyzikální fakulta} & -- \\
		2020-2023 & Informatika & \href{https://www.mff.cuni.cz/}{Univerzita Karlova, Matematicko-Fyzikální fakulta} & Titul Bc. \\
		2012-2020 & Osmileté gymnázium & \href{https://www.gytool.cz/}{Gymnázium Olomouc-Hejčín} & Maturita \\
		2007-2012 & Základní škola & \href{https://zsstupkova.cz/}{ZŠ Stupkova, Olomouc} & \\
	\end{tabular}
\end{table}

\subsection*{Projekty na kterých jsem pracoval}

\begin{itemize}
	\item Aplikace \href{https://rodoc-app.github.io/}{rodoc} pro správu rodokmenu, napsaná v \texttt{C++} s pomocí \texttt{Qt}. Tento projekt byl součástí mé bakalářské prace.
	\item Dalším menším projektem je moje řešení \href{https://adventofcode.com/}{advent of code}, které obsahuje i skript napsaný v jazyce \texttt{Perl} pro autoamtické vytváření projektů a stránek -- \href{https://github.com/metury/advent-of-code}{github repositář}.
\end{itemize}

\subsection*{Pracovní zkušenosti}

\begin{table}[!ht]\centering
	\begin{tabular}{p{20mm} p{150mm}}
		\textit{Roky} & \textit{Práce} \\
		2019 a 2020 & Letní manuální brigáda v Dánsku -- třídění a příprava balíčků v logistické firmě Prime Cargo.
	\end{tabular}
\end{table}

\subsection*{Osobní vlastnosti}

\begin{itemize}
	\item Mile rád se naučím používat nové technologie a postupy, protože ve světě IT není nic stálé.
	\item Vymýšlení řešení problémů, které se vyskytují na denní bázi, ať už v práci nebo v životě.
	\item Považuji se za pracovitého člověka, který je ochotný a rád pomůže ostatním.
	\item Mimo to se rád pohybuji ve společnosti dalších lidí. A také si potrpím na dochvilnosti.
	\item Jelikož nikdo není dokonalý, tak i já mám nevýhody, jako je třeba občasná tvrdohlavost.
\end{itemize}

\subsection*{Osobní zájmy a koníčky}

\begin{itemize}
	\item Několik let (zhruba 8) jsem hrál závodně florbal a i nadále, když se vyskytne příležitost, tak si zahraji.
	\item Hra na příčnou flétnu. Naučil jsem se při navštěvování \href{https://www.zus-zerotin.cz/}{ZUŠ Žerotín}, kde jsem mimo sólovou hru hrál i v komorních uskupeních a také sborech, s kterými jsme mimo jiné soutěžili v zahraničích přehlídkách.
\end{itemize}

\end{document}