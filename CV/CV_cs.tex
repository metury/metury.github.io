\documentclass[12pt,a4paper]{article}
\setlength\textwidth{160mm}
\setlength\textheight{247mm}
\setlength\oddsidemargin{0mm}
\setlength\evensidemargin{0mm}
\setlength\topmargin{0mm}
\setlength\headsep{0mm}
\setlength\headheight{0mm}
\let\openright=\clearpage

\usepackage{hyperref}

\pagestyle{empty}

\begin{document}

\begin{center}
	\section*{Životopis}
\end{center}

\subsection*{Základní informace}

\begin{itemize}
	\item Jméno: \textit{Tomáš Turek}
	\item Datum narození: \textit{24.03.2001}
	\item Původ: \textit{Olomouc, Česká republika}
	\item Mail: \href{mailto:tom.turek@atlas.cz}{\textit{tom.turek@atlas.cz}}
	\item Github: \href{https://github.com/metury}{\textit{metury}}
\end{itemize}

\subsection*{Dosavadní studium}

\begin{itemize}
	\item Základní škola (1.-5. ročník): \href{https://zsstupkova.cz/}{ZŠ Stupkova, Olomouc}
	\item Gymnázium (osmileté s dokončenou maturitou): \href{https://www.gytool.cz/}{Gymnázium Olomouc-Hejčín}
	\item Bakalářské studium v oboru Informatika: \href{https://www.mff.cuni.cz/}{Univerzita Karlova, Matematicko-Fyzikální fakulta}
	\item Magisterské studium (zatím nedokončené) v oboru Diskrétní modely a algoritmy: \href{https://www.mff.cuni.cz/}{Univerzita Karlova, Matematicko-Fyzikální fakulta}
\end{itemize}

\subsection*{Pracovní zkušenosti}

\begin{itemize}
\item Manuální brigáda v Dánsku. - Sice není ničím zajímavá z pohledu programátorských zkušeností, ale i tak vhodné zmínit práci v cizí zemi a cizím jazykem.
\end{itemize}

\subsection*{Osobní vlastnosti}

Co se týče lingvistiky, tak Český jazyk je samozřejmostí a k tomu solidní znalost \textbf{Anglického jazyka }a základní znalost \textbf{Německého jazyka}.

Ohledně jazyků, ale už ne lingvistických mám znalost následujících jazyků:

\begin{itemize}
	\item Pokročilá znalost \textbf{C++}.
	\item Znalost jazyka \textbf{Java}.
	\item Znalost jazyka \textbf{Python}.
	\item Základní znalost \textbf{C\#}.
	\item Základní znalost exotičtějšího funkcionálního jazyka \textbf{Haskell}.
\end{itemize}

Nicméně se nebráním žádným jiným jazykům, protože dle mé filosofie je třeba znát obecné postup samozřejmostí ay a konkrétní znalost specifického jazyka je jen něco podobného "dialektům".

Mezi své silnější stránky bych dal pracovitost a také vstřícnost a ochota pomoci ostatním. Stejně tak se nevyhíbám osobním kontaktům a rád komunikuji a řeším problémy. Další v pořadí je moje dochvilnost.

Nikdo není dokonalý a to samozřejmě znamená, že mám své neduhy. Jedním z nich je, že pokud si myslím a jsem přesvědčený, že mám pravdu, tak nechci ukončit debatu tím, že se neshodneme (ale pokud protistrana dá dobré protiargumenty, tak jsme rychle schopný uznat, že jsme se mýlil). Určitě mám více chyb, ale buď je úmyslně přehlížím anebo se je snažím eliminovat.

\subsection*{Zájmy a preference}

Osobně preferuji pracovat na zařízeních s operačním systémem \textbf{Linux}, ale nevyhnu se ani práci na počítači mající\textbf{Windows}. S Linuxem se také pojí, že jsem fanouškem filosofie open-source aplikací.

Spíše preferuji řešit problémy a navrhovat jejich řešení, než je "datlovat" do počítače. Kvůli tomu jsem si vybral Matfyz na rozdíl od jiných českých univerzit, které dokáží "vygenerovat" lepší developery, ale nemají tak velkou znalost matematiky a informatické teorie. S tím také zmíním, že ještě mám v plánu pokračovat magisterským studiem teoretické informatiky.

\end{document}