\documentclass{article}

\usepackage{hyperref}

\pagestyle{empty}

\begin{document}

\begin{center}
\section*{Resumé}
\end{center}

\subsection*{General information}

\begin{itemize}
	\item Name: \textit{Tomáš Turek}
	\item Date of birth: \textit{24.03.2001}
	\item Place: \textit{Olomouc, Česká republika}
	\item Mail: \href{mailto:tom.turek@atlas.cz}{tom.turek@atlas.cz}
	\item Github: \href{https://github.com/metury}{metury}
\end{itemize}

\subsection*{Previous studies}

\begin{itemize}
	\item Primary school (first 5 years): \href{https://zsstupkova.cz/}{ZŠ Stupkova, Olomouc}
	\item High school (eight years with a finished diploma): \href{https://www.gytool.cz/}{Gymnázium Olomouc-Hejčín}
	\item Bachelors studies (yet not finished) in Computer science: \href{https://www.mff.cuni.cz/}{Univerzita Karlova, Matematicko-Fyzikální fakulta}
\end{itemize}

\subsection*{Previous experiences}

\begin{itemize}
	\item Manual side job in Denmark. - This job was not one of the most important for progtramming job, but it may be useful to mention it anyway since it means I had to live and work in other state for some time.
\end{itemize}

\subsection*{Personal characteristics}

I am fluent in Czech. Then I have a solid knowledge of \textbf{English} and some basic or so called beginner-level knowledge of \textbf{German}.

Speaking of languages I will mention programming languages:

\begin{itemize}
	\item Advanced knowledge of \textbf{C++}.
	\item Standard knowledge of \textbf{Java}.
	\item Also standard knowledge of \textbf{Python}.
	\item Basic knowledge of \textbf{C\#}.
	\item Basic knowledge of a functional language \textbf{Haskell}.
\end{itemize}

But I would state that I think of separate languages as a "dialects" and thus the theoretical knowledge of computer science and mathematics in general is far more important than learning specific languages which also change during the time.

Mezi své silnější stránky bych dal pracovitost a také vstříctnost a ochota pomoci ostatním. Stejně tak se nevyhíbám osobním kontaktům a rád komunikuji a řeším problémy. Další v pořadí je moje dochvilnost.

Nikdo není dokonalý a to samozřejmě znamená, že mám své neduhy. Jedním z nich je, že pokud si myslím a jsem přesvědčený, že mám pravdu, tak nechci ukončit debatu tím, že se neshodnem (ale pokud protistrana dá dobré protiargumenty, tak jsme rychle schopný uznat, že jsme se mýlil). Určitě mám více chyb, ale buď je úmyslně přehlížím anebo se je snažím eliminovat.

\subsection*{Zájmy a preference}

Osobně preferuji pracovat na zařízeních s operačním systémem \textbf{Linux}, ale nevyhnu se ani práci na počítačí mající \textbf{Windows}. S Linuxem se také pojí, že jsem fanouškem filosofie open-source aplikací.

Spíše preferuji řešit problémy a navrhovat jejich řešení, než je "datlovat" do počítače. Kvůli tomu jsem si vybral Matfyz na rozdíl od jiných českých univerzit, které dokáží "vygenerovat" lepší developery, ale nemají tak velkou znalost matematiky a informatické teorie.

S tím také zmíním, že ještě mám v plánu pokračovat magisterským studiem teoretické informatiky.

\end{document}
