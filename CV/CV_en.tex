\documentclass{article}

\usepackage{hyperref}

\pagestyle{empty}

\begin{document}

\begin{center}
\section*{Resumé}
\end{center}

\subsection*{General information}

\begin{itemize}
	\item Name: \textit{Tomáš Turek}
	\item Date of birth: \textit{24.03.2001}
	\item Place of birth: \textit{Olomouc, Česká republika}
	\item Mail: \href{mailto:tom.turek@atlas.cz}{tom.turek@atlas.cz}
	\item Github: \href{https://github.com/metury}{metury}
\end{itemize}

\subsection*{Previous studies}

\begin{itemize}
	\item Primary school (first 5 years): \href{https://zsstupkova.cz/}{ZŠ Stupkova, Olomouc}
	\item High school (eight years with a finished diploma): \href{https://www.gytool.cz/}{Gymnázium Olomouc-Hejčín}
	\item Bachelors studies in Computer science: \href{https://www.mff.cuni.cz/}{Univerzita Karlova, Matematicko-Fyzikální fakulta}
	\item Master studies (yet not finished) in Discrete models and algorithms: \href{https://www.mff.cuni.cz/}{Univerzita Karlova, Matematicko-Fyzikální fakulta}
\end{itemize}

\subsection*{Previous experiences}

\begin{itemize}
	\item Manual side job in Denmark. - This job was not one of the most important for progtramming job, but it may be useful to mention it anyway since it means I had to live and work in other state for some time.
\end{itemize}

\subsection*{Personal characteristics}

I am fluent in Czech. Then I have a solid knowledge of \textbf{English} and some basic or so called beginner-level knowledge of \textbf{German}.

Speaking of languages I will mention programming languages:

\begin{itemize}
	\item Advanced knowledge of \textbf{C++}.
	\item Standard knowledge of \textbf{Java}.
	\item Also standard knowledge of \textbf{Python}.
	\item Basic knowledge of \textbf{C\#}.
	\item Basic knowledge of a functional language \textbf{Haskell}.
\end{itemize}

But I would state that I think of separate languages as a "dialects" and thus the theoretical knowledge of computer science and mathematics in general is far more important than learning specific languages which also change during the time.

Among my strengths I would put hardworking and also helpfulness and willingness to help others. Likewise, I don't shy away from personal contacts and I like to communicate and solve problems. Next in order is my punctuality.

No one is perfect and of course that means I have my flaws. One of them is that if I think and believe I'm right, I don't want to end the debate by disagreeing (but if the other side gives good counter-arguments, I am quickly able to acknowledge that I was wrong). I'm sure I have multiple flaws, but I either deliberately overlook them or try to eliminate them.

\subsection*{Interests and preferences}

Personally, I prefer to work on devices running \textbf{Linux}, but I can't avoid working on computers running \textbf{Windows}. Linux is also linked to the fact that I am a fan of the open-source application philosophy.

I prefer to solve problems and design solutions rather than "date" them into the computer. This is why I chose Matfyz, unlike other Czech universities that can "generate" better developers, but do not have such a great knowledge of mathematics and computer theory.

With that said, I will also mention that I still plan to pursue a master's degree in theoretical computer science.

\end{document}
